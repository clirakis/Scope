\documentclass[a4paper]{book}
\usepackage{a4wide}
\usepackage{makeidx}
\usepackage{fancyhdr}
\usepackage{graphicx}
\usepackage{multicol}
\usepackage{float}
\usepackage{textcomp}
\usepackage{alltt}
\title{Superconducting Notes}
\date{31-Mar-06}
\author{C.B. Lirakis}
\begin{document}

The solution for a driven SHO is given by a particular and complementary 
solution. The complemntary solution is to the homogenious equation where there
is no driving force. It also represents the transient portion of the solution.

The complemntary solution to a non-driven but damped SHO is:\hfill\break
$x_c(t) = e^{\beta t} [A_1 e^{\sqrt{\beta^2-\omega_0^2}t}+A_2 e^{-\sqrt{\beta^2-\omega_0^2}t]}$\hfill\break

\begin{enumerate}
\item Underdamped $\omega_0^2 > \beta^2$
\item Critically Damped $\omega_0^2 = \beta^2$
\item Underdamped $\omega_0^2 < \beta^2$ (imaginary solutions)
\end{enumerate}


So for the undriven case the frequency is given by: \hfill\break
$\omega_1 = \sqrt{\omega_0^2-\beta^2}$ \hfill\break

So now the trick is from observing the data, what are the individual elements. 
We can extract the wave motion from the heave, perhaps. The Natural frequency 
of the boat can be extracted from the roll and pitch data. 

The equation of motion for a SHO driven by a harmonic force is: \hfill\break
The full equation for a sinusoidally driven oscillation is: \hfill\break
$x(t) = x_c(t) + x_p(t) $
where \hfill \break
$x_p(t) = Dcos(\omega t - \delta)$

Solving for the details by putting this into the equation of motion gives: \hfill\break
$D={A\over{\sqrt{(\omega_o^2 - \omega^2)^2cos\delta+2\omega\beta sin\delta}}}$ \hfill\break
$D={A\over{\sqrt{(\omega_o^2 - \omega^2)^2+4\omega^2\beta^2}}}$\hfill\break
\begin{enumerate}
\item $\omega$ is the driving frequency.
\item $\omega_0$ is the natural frequency
\item $\beta$ is the damping factor
\item A  is the amplitude of the driving force. 
\end{enumerate}

The specific frequency in the driven case is: \hfill\break
$\omega_R = \sqrt{\omega_0^2-2\beta^2}$ \hfill\break
and the quality factor is given by: \hfill\break
$Q={\omega_R \over{2\beta}} \approx {\omega_0\over{\Delta\omega}}$


The model I have in mind is that of a long half cylinder. I need to look at 
the intertia along the z axis (pitch) and $\rho$ for the roll. I'm not sure 
how to model the heading. 

One thing I did think about is that really the number for the heading should 
just be continious. Never treat it as modulo. Only convert it to being bound 
by 0-$2\pi$ after the fact. 

Another thought I had was for the process noise in the Kalman filter. I have 
always thought of this as being fixed, but there is no reason that it too
can't be a function. Especially in the case where no driving force is applied
and the noise can be $e^{-\beta t}$ or even positive for growing errors. 
\end{document}
